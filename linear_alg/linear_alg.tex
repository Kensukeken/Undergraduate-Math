\documentclass{article}
\usepackage{amsmath, amssymb}

\title{Linear Algebra Course Notes}
\author{Kensukeken}
\date{June 28th, 2024}

\begin{document}

\maketitle

\section{Introduction}

\subsection{Definitions and Notations}

In linear algebra, we deal with scalars, vectors, and matrices. Scalars are single numbers, vectors are ordered lists of numbers, and matrices are rectangular arrays of numbers.

\subsection{Scalars, Vectors, and Matrices}

A scalar is a single number, usually denoted by a lowercase letter (e.g., \(a, b, c\)). A vector is an ordered list of numbers, denoted by a lowercase bold letter (e.g., \(\mathbf{v}\)). A matrix is a rectangular array of numbers, denoted by an uppercase letter (e.g., \(A, B, C\)).

\section{Systems of Linear Equations}

\subsection{Row Reduction and Echelon Forms}

To solve systems of linear equations, we use row reduction to transform the augmented matrix into row echelon form.

\[
\begin{array}{ccc|c}
1 & 2 & 3 & 4 \\
0 & 1 & 4 & 5 \\
0 & 0 & 2 & 6 \\
\end{array}
\]

\subsection{Solutions to Linear Systems}

A system of linear equations can have no solution, exactly one solution, or infinitely many solutions. This can be determined by the row echelon form of the augmented matrix.

\section{Matrix Algebra}

\subsection{Matrix Operations}

Matrices can be added, subtracted, and multiplied, and they can also be multiplied by scalars.

\[
A = \begin{pmatrix}
1 & 2 \\
3 & 4
\end{pmatrix}, \quad B = \begin{pmatrix}
5 & 6 \\
7 & 8
\end{pmatrix}
\]

\[
A + B = \begin{pmatrix}
6 & 8 \\
10 & 12
\end{pmatrix}
\]

\subsection{Inverses and Transposes}

The transpose of a matrix \(A\) is denoted \(A^T\). A matrix \(A\) is invertible if there exists a matrix \(B\) such that \(AB = BA = I\).

\section{Determinants}

\subsection{Properties of Determinants}

The determinant is a scalar value that can be computed from a square matrix.

\[
\det(A) = a_{11}a_{22} - a_{12}a_{21}
\]

\subsection{Cofactor Expansion}

The determinant can be computed using cofactor expansion.

\section{Vector Spaces}

\subsection{Definitions and Examples}

A vector space is a collection of vectors that can be added together and multiplied by scalars.

\subsection{Subspaces, Basis, and Dimension}

A subspace is a subset of a vector space that is also a vector space. A basis is a set of vectors that span the vector space, and the dimension is the number of vectors in the basis.

\section{Eigenvalues and Eigenvectors}

\subsection{Characteristic Equation}

The eigenvalues of a matrix \(A\) are found by solving the characteristic equation \(\det(A - \lambda I) = 0\).

\subsection{Diagonalization}

A matrix is diagonalizable if it has enough eigenvectors to form a basis.

\section{Orthogonality}

\subsection{Inner Product, Norm, and Orthogonality}

The inner product of two vectors is a scalar. The norm of a vector is its length, and two vectors are orthogonal if their inner product is zero.

\subsection{Gram-Schmidt Process}

The Gram-Schmidt process is a method for orthogonalizing a set of vectors in an inner product space.

\section{Applications}

\subsection{Linear Transformations}

A linear transformation is a mapping between vector spaces that preserves addition and scalar multiplication.

\subsection{Applications in Differential Equations}

Linear algebra techniques are used to solve systems of differential equations.

\end{document}
